
%TODO w kazdy rozdziale którki wstęp co się w nim będzie znajdować
%TODO sprawdzic czy we wstępie są cele pracy - szybkie protypowane i zastsowania tworzonego modelu, ewkuacjia ludzi itd
% We wstępie TEZA + dokładniejszy opis kolejnyc rozdziałów
%TODO pozamieniać pogrubienia na pochylenia. pogrubienia głupio wyglądają. sprw. jak ma dr.Wąs u siebie

\documentclass[pdflatex,11pt]{aghdpl}
% \documentclass{aghdpl}               % przy kompilacji programem latex
% \documentclass[pdflatex,en]{aghdpl}  % praca w j?zyku angielskim
\usepackage[polish]{babel}
\usepackage[utf8]{inputenc}

% dodatkowe pakiety
\usepackage{enumerate}
\usepackage{listings}
\lstloadlanguages{TeX}



%---------------------------------------------------------------------------

\author{Dorota Wojtałow, Jacek Złydach}
\shortauthor{D. Wojtałów, J. Złydach}
%\shortauthor{M. Szpyrka}

\titlePL{Symulacja rozprzestrzeniania się dymu i ognia w oparciu o niehomogeniczne automaty komórkowe}
\titleEN{Simulation of fire and smoke by using non-homogeous cellular automata}
%\titleEN{Thesis in \LaTeX}


\thesistypePL{Praca inżynierska}

\supervisorPL{dr inż. Jarosław Wąs}
\supervisorEN{Jarosław Wąs Ph.D}

\date{2010}

\facultyPL{Wydział Elektrotechniki, Automatyki, Informatyki i Elektroniki}

\acknowledgements{Serdecznie dziękujemy \dots }

%---------------------------------------------------------------------------

\begin{document}

\titlepages

\tableofcontents
\clearpage

\chapter{Wstęp}
\label{cha:wstep}

%wg. strong Wojnickiego 5 pktow które powinny znaleźć się we wstępie:
% 1. Co - przediot, problem pracy
% 2. Jak - metoda, krotko
% 3. Dlaczego - źrodla problemu badawczego
% 4. Po co - implikacje, konsekwencje, walory
% 5. Co w kolejnych rozdziałach
\section{Temat pracy} % 1 Co

Tematem pracy jest stworzenie symulacji rozprzestrzeniania się dymu i ognia 
w opraciu o niehomogeniczne automaty komórkowe. Zakres pracy obejmuje stworzenie symulacji rozchodzenia się dymu i ognia na podstawie automatów komórkowych wraz z jej wizualizacją, a także walidację stworzonego modelu. 
Celem pracy jest pokazanie możliwości niehomogenicznych automatów komórkowych jako narzędzia umożliwiającego 
odzwierciedlenie rzeczywistego rozprzestrzeniania się dymu i ognia podczas pożaru. 

\section {Geneza tematu} %3 i 4 - Dlaczego i po co

W dobie wszechobecnej urbanizacji i ciągłego budownictwa, wraz ze wzrostem świadomości dotyczącej bezpieczeństwa
pożarowego oraz zaangażowania w jego zagwarantowaniu pojawiła się potrzeba możliwości modelowania i obserwacji rozprzestrzeniania się ognia w zamkniętych budynkach.
Wspomniane symulacje pożarów wykazują szereg zastosowań. Są z powodzeniem wykorzystywane w śledztwach. Dają możliwość odtworzenia przebiegu zdarzeń i porównania z wynikami oględzin. Umożliwiają
zbadanie prototypu budynku pod kątem gwarancji bezpieczeństwa pożarowego. 
Ułatwiają projektowanie systemów oddymiania. W połączeniu z modelami ewakuacji ludzi
stanowią kompleksowy system ułatwiający tworzenie bezpiecznych budowli.

W ostatnich latach powstał szereg programów umożliwiających wizualizcję symulacji rozchodzenia ognia. Opracowane dotychczas rozwiązania swoje działanie 
opierają na metodach numerycznej dynamiki płynów (ang. Computational Fluid Dynamics). Niewątpliwą zaletą numerycznego podejścia jest dokładność wyników. 
Głównymi wadami jest złożoność obliczeń i stopień komplikacji modelu. 
Niehomogeniczne automaty komórkowe umożliwiają znaczne uproszczenie modelu. Uproszczenie modelu powoduje z kolei redukcję złożoności
obliczeń czyniąc automaty komórkowe szczególnie dogodną metodą w przypadku tworzenia prototypów oraz symulacji czasu rzeczywistego.

%dopisać coś jeszcze o tym, że nie ma takich symulatorów wyk. automaty komórkowe 

\section {Realizacja projektu} % 2 - jak
Praca została zrealizowana jako wolnostojąca aplikacja komputerowa napisana w języku Java. Do renderowania grafiki trójwymiarowej
zostala użyta biblioteka graficzna Java3D. Aplikacja została przetestowana z wykorzystaniem biblioteki JUnit4. 

\section{Struktura pracy} % 5 - co w kolejnych rozdziałach Tutaj na pewno trzeba dać odnośniki do rozdziałów a nie tytuły!
Praca składa się z [iluś] rozdziałów. 
W pierwszym rozdziale znajdują się podstawy teoretyczne, związane zarówno z modelowanymi zjawiskami fizycznymi jak i użytym algorytmem.
Rozdział Modele symulacji zawiera propozycje zweryfikowanych modeli rozprzestrzeniania się dymu i ognia zaprojektowanych w oparciu
o niehomogeniczne automaty komórkowe. Rozdział Implementacja przedstawia sposób realizacji projektu, napotkane problemy oraz 
ich rozwiązania. Opisuje możliwości graficznego interfejsu użytkownika oraz sposób korzystania z niego.
 % W podsumowaniu znajduje się podsumowanie.... nie wiem jeszcze jak to ładnie ująć
%cos wiecej o rozdzialach

%TODO:
%Wstawić rysunek konwekcji narysowany ze screena z Rosenbajgerowymi strzałkami
%Konwekcja, a grawitacja i prawo Archimedesa
\chapter{Teoria}
\label{cha:Teoria}
Kluczowym elementem, niezbędnym do prawidłowego zamodelowania pożaru
jest zrozumienie czym jest ogien oraz poznanie zjawisk jakim podlega. Niniejszy rozdział zawiera
krótki wstęp teoretyczny, przedstawiający zjawiska fizyczne niezbędne do zrozumienia istoty 
pożaru i prawidłowego jego zamodelowania.
\section {Czym jest ogień}
Ogień nie jest substancją.
Ogień powstaje jako produkt reakcji chemicznej zachodzącej między paliwem i tlenem.
Obserwowalną postać ognia, czyli to co widzimy i nazywamy ogniem tworzy światło powstałe w wyniku ruchu rozgrzanego powietrzna.
Jest to jednak tylko jeden z aspektów tego złożonego procesu.
Elementami koniecznymi do powstania i podtrzymania ognia są:
\begin{itemize}
\item tlen
\item paliwo
\item ciepło
\end{itemize}


Paliwem może być ciecz, ciało stałe lub gaz. Samo w sobe paliwo nie ulega spalaniu. 
Paliwo pod wpływem ciepła pochodzącego np. z zapałki lub otrzymanego od innego nagrzanego ciała ogrzewa się. Po osiągnięciu
odpowiedniej temperatury, paliwo ulega procesowi dekompozycji. Jednym z produktów dekompozycji
są opary. W przypadku jednego z najbardziej popularnych paliw - drewna oprócz oparów w wyniku dekompozycji otrzymujemy węgiel i popiół.
Spalanie drewna przedstawia poniższa reakcja \ref{reakcja_spalania}:
\begin {equation}
CH_20+0_2+heat ->C0_2 + C0+ C + N_2 + H_20
\label {reakcja_spalania}
\end {equation}
Kiedy opary osiągną odpowiednią temperaturę tzw. temperaturę zapłonu (w przypadku drewna wynosi ona ok. $300^\circ C$) oraz gdy ich stężenie
w powietrzu jest odpowiednie może dojść do zapłonu. Do zapłonu dochodzi w wyniku kontaktu z otwartym ogniem, iskrą lub w wyniku osiągnięcia
przez opary temperatury tzn. samozapłonu. Wynikiem zapłonu jest spalanie oparów. Jak widać głównym substratem reakcji spalania są opary powtające w wyniku
ogrzania paliwa. W przypadku niektórych paliw jest to jedyny reagent. Jednym z przykładów jest benzyna, która w wyniku ogrzania w całości zamienia się w opary ulegające spalaniu. W przypadku drewna, poza oparami spalaniu ulega także węgiel. Jest to jednak reakcja bardzo powolna.
Na szczególną uwagę zasługuje fakt wytwarzania energii cieplnej w procesie spalania, co powoduje samoistne podtrzymanie ognia. Płomień będacy
wizualną postacią spalania ogrzewa sąsiadujące cząsteczki paliwa, dzięki czemu nie gaśnie.

Bardzo istotnym reagentem w procesie spalania jest tlen. Atomy gazów, oparów powstałych w wyniku podgrzania paliwa w wyniku zapłonu łączą się z tlenem.
Aby mogło dojść do reakcji spalania bardzo ważne jest zachowanie odpwiednich proporcji między substratami reakcji. Lower Explosive Limit (LEL) określa 
minimalne stężenie oparów w powietrzu, konieczne aby mogło dojść do zapłonu. Odpowiednio, Upper Explosive Limit (UEL) oznacza maksymalne stężenie oparów, powyżej
którego nie dojdzie do zapłonu. Przykładowo: dla tlenku węgla $LEL=12$, natomiast  $UEL=75$, co oznacza w powietrzu musi być między $12\%-75\%$ aby mogło dojść
do jego zapalenia. $LEL$ oraz $UEL$ określają także pośrednio wymaganą ilość tlenu. Dla większości paliw ilość tlenu wymaganego do zapłnu wynosi ok. $15\%$. 

\section {Propagacja ciepła}
Jak zostało wspomniane w rozdziale \ref{Proces spalania} jednym z czynników niezbędnych
do podtrzymania ognia jest cieło. Ciepło podczas pożaru jest propagowane na trzy różne sposoby:
\begin {itemize}
\item Przewodnictwo
\item Konwekcja
\item Radiacja
\end {itemize}

\subsection {Przewodnictwo}
\label{Przewodnictwo}
 Przewodnictwo cieplne jest procesem, który polega  na wymianie ciepła 
pomiędzy nierównomiernie ogrzanymi ciałami będącymi w kontakcie. Zachodzi ono we wszystkich stanach skupienia: ciałach stałych, cieczach i gazach, jednak sposób i skala tego zjawiska jest bardzo zróżnicowana. Najczęściej mówimy o przewodnictwie w ciałach stałych.
W cieczach i gazach występuje ono niezmiernie rzadko i polega na zderzeniach cząsteczek podlegających
niezorganizowanym, przypadkowym ruchom i ich dyfuzji.
W ciałach stałych przenoszenie ciepła odbywa się  na dwa sposoby:
\begin{itemize}
\item dzięki drganiom atomów
\item poprzez ruch elektronów
\end {itemize}
Celem omawianego przewodnictwa jest 
osiągnięcie równowagi cieplnej. Podczas przewodnictwa ciepło jest zawsze przenoszone od
ciała o większej temperaturze do ciała o niższej. Zgodnie z zasadą zachowania energi, głoszącą że w układzie 
izolowanym suma wszystkich energii jest stała, ilość energii uzyskanej przez ciało chłodniejsze jest równa
ilości energii oddanej przez cieplejszy obiekt. Energia przenoszona jest wraz z ruchem cząsteczek wewnętrznych.
Nie wszystkie ciała przewodzą ciepło w takim sam sposób.
Zależność między ilością ciepła przewodzonego przez ciało, a jego zmianą temperatury najlepiej opisuje prawo Fouriera.
Przyjmuje ono następującą postać:
\begin{equation}
 q(r,t)=-k*grad T
 \label{eqn:fourier}
\end {equation}
gdzie:
k - współczynnik przewodzenia ciepła $[W / (m*K)]$
T - temperatura $[K]$
q - natężenie strumienia ciepła  $[W/(m^2)]$
Prawo Fouriera oznacza, że gęstość strumienia ciepła przekazywana w jednostce czasu przez jednostkową powierzchnię 
jest proporcjonalna do gradientu temperatury. Minus we wzorze wynika ze wspomianego wyżej kierunku przepływu ciepła:
od ciała cieplejszego do zimniejszego. Strumień ciepła jest mierzony w kierunku zgodnym z jego przepływem, zatem przyrost
temperatury będzie miał wartość ujemną.


Do dobrych przewodków należą przede wszystkim:
\begin {itemize}
\item metale - do najlepszych należą srebro, miedź, złoto, aluminium
\end {itemize}
Źle przewodzą ciepło:
\begin {itemize}
\item drewno
\item papier
\item ciecze
\item gazy
\end {itemize}
Zła przewodność cieczy i gazów wynika z istoty procesu przewodnictwa w tych stanach skupienia. Za wysoką wartość 
współczynnika przewodnictwa odpowiada ruch elektronów. Dlatego też, we wszystkich dialektrykach wartość ta
przyjmuje wartości z przedziału $[0,001-3][W/(m*K)]$, podczas gdy w metalach może siegać ona nawet $ 400 [W/m*k]$

Na uwagę zasługuje też fakt, że przewodność metali maleje wraz ze wzrostem ich temperatury.
Tabela \ref {przewodnictwa} zawiera współczynniki przewodnictwa przykładowych materiałów, które zostały wykorzystane przy
testowaniu algorytmu symulacji.
%http://tabelechemiczne.chemicalforum.eu/przewodnictwo_ciala.html
%Dodać do bibliografii tablice fizyczne z których to wzięte
\begin{table}
\begin {center}
\begin{tabular} {|l | c | c|}
\hline
Materiał & Temp. $[C]$ & Wspł. przewodnictwa $[W/(m*K)]$ \\ \hline
Beton & 20 & 0.84-1.3  \\ \hline
Drewno & - & 0.1-0.17  \\ \hline
Szklo crown & 20 & 0.22-0.29  \\ \hline
Azbest & 20 & 0.16-0.37 \\ \hline
Guma wulkanizowana & 20 & 0.22-0.29 \\ \hline
Miedź & 20 & 400 \\ \hline
Stal & 20 & 10 \\ \hline
Ołów & 20 & 30 \\ \hline
Powietrze & 20 & 0.025 \\ 
\hline
\end {tabular}
\caption{Współczynniki przewodnictwa materiałów}
\label{przewodnictwa}
\end{center}
\end {table}
\subsection{Konwekcja}
\label{Konwekcja}
Konwekcja, zwana też unoszeniem lub wnikaniem jest to zgodnie ze szkolną definicją sposób przewodnictwa ciepła polegający na
"unoszeniu prabranej energii cieplnej przez cząsteczki substancji i dzięki swojej wędrówce przekazywaniu
energii innym cząsteczkom". 
ak się znajdzie jakaś inna przystęna definicja to podmienić
Konwekcja zachodzi we wszystkich płynach, czyli zarówno ciecach jak i gazach. Nie zachodzi natomiast w ciałach stałych.
Konwekcja ze względu na połączenie w sobie dwóch zjawisk: 
\begin{itemize}
\item przekazywania ciepła
\item ruchu płynów
\end{itemize}
jest zjawiskiem niezwykle skomplikowanym do teoretycznego ujęcia. Przenoszenie ciepła w konwekcji zachodzi 
wskutek ruchu płynu, tak więc warunkiem niezbędnym do wystąpienia zjawiska konwekcji jest ruch ośrodka.
Można wyróżnić dwa podstawowe typy konwekcji, dzielące zjawisko wnikania ze względu na przyczynę ruchu ośrodka:
\begin{itemize}
\item konwekcja naturalna - w tym przypadku ruch płynu wywołany jest różnicami gęstości substancji znajdujących się w polu grawitacyjnym
\item konwekcja wymuszona - ruch płynu spowodowany jest działaniem urządzeń zewnętrznych (wentylatorów, pomp)
\end {itemize}
Przykładem konwekcji naturalnej jest unoszenie ciepłego powietrzna w pomieszczeniu. 
Ogrzane powietrze zmniejsza swoją gęstość, w wyniku czego unosi się do góry. Jego miejsce wypełnia zimne powietrze, 
które w kolejnym etapie ulega ogrzaniu rozpoczynając kolejny cykl wędrówki powietrza. Ruchy powietrza wywołane zjawiskiem 
konwekcji tworzą tzw. prądy konwekcyjne.
%TODO dodac rysunek konwekcji
 Konwekcja naturalna jest typem konwekcji występującym podczas pożaru. 

\subsection {Radiacja}
\label{Radiacja}
Radiacja, czyli inaczej promieniowanie jest to sposób rozchodzenia ciepła w postaci fal elektromagnetycznych. 
Najważniejszym aspektem przewodnictwa ciepła przez promieniowanie jest możliwość wymiany ciepła między ciałami
nie stykającymi się.
W bardzo niskich temperaturach ilość przekazywanego przy pomocy radiacji ciepła jest tak mała, że zjawisko to może być pomijane.
Wzrost znaczenia promieniowania następuje wraz ze wzrostem temperatury ciał wymieniających ciepło.
Przyjmuje się, że radiacja zachodzi dla ciał o temperaturach wyższych od $0 ^\circ$ Kelvin.
Promieniowanie jest rodzajem wymiany energii, która nie wymaga żandego nośnika. Każde ciało emituje fale.
W normalnych warunkach większość promieniowania zachodzi przy udziale fal podczerwonych. Należy jednak pamiętać
że w radiacji mogą brać udział także fale świetlne czy ultrafioletowe. Poza emisją promieniowania każde ciało reaguje także
na fale wysyłane przez innych. Dla każdego ciała jesteśmy w stanie określić wartości trzech współczynników opisujących 
reakcję ciała na wiązkę promieniowania. Należą do nich:
\begin {itemize}
\item Absorpcyjność czyli pochłanialność
\item Refleksyjność czyli odbijalność
\item Przepuszczalność
\end {itemize}
Radiacja następuje we wszystkich kierunkach aż  do momentu zablokowania drogi promieni przez ciało pochłaniające je.
Większość ciał stałych o rozmiarach większych od kilku mikrometrów nie przepuszcza promieniowania. Na wspomnianej 
głębokości pod powierzchnią ciała następuje całkowita absorpcja promieniowania cieplnego. Ponadto ciała stałe mogą przepuszczać
fale tylko o określonej długości. Przykładem jest szkło, które przepuszcza jedynie fale świetlne.
Ilość promieniowania emitowanego przez ciała szare można obliczyć ze wzoru \ref{emisyjnosc}
\begin {equation}
\dot{Q_{emit}}=\sigma*\varepsilon*A_{s}*T_{s}^4
\label {emisyjnosc}
\end {equation}
gdzie
\begin{itemize}
\item $\varepsilon \in (0,1)$ - emisyjność powierzchni. $\varepsilon=1$ - dla ciała doskonale czarnego. Określa jak bardzo dane 
ciało jest podobne do ciała doskonale czarnego.
\item $\sigma = 5.67 * 10^-8 [W/(m^2 * K^4]) $ - stała promieniowania
\item $A_{s} [m^2]$ - powierzchnia 
\item $T_{s} [K]$ - temperatura
\end {itemize}
Przeanalizujmy przykład promieniowania między rzeczywistymi obiektami znajdującymi się w pewnym pomieszczeniu np. stół w pokoju.
W przypadku jednej powierzchni zamkniętej w innej (w omawianym przypadku wewnętrzną powierzchnią będzie powierzchnia stołu, natomiast zewnętrzną ściany pokoju) zakłada się, że wewnętrzna powierzchnia "nie opromieniowuje samej siebie".
Innymi słowy całe promieniowania ciała wewnętrznego przechodzi do powierzchni zewnętrznej. W drugim kierunku następuje tylko częściowe
przejście energii z ciała zewnętrznego do wewnątrz. Ponadto, w przypadku gdy otaczająca powierzchnia jest znacząco większa od powierzchni wewnętrznej i obie powierzchnie są oddzielone gazem, który nie promieniuje (powietrze) zjawisko promieniowania zachodzi
równolegle ze zjawiskiem konwekcji i oba te zjawiska należy wziąć pod uwagę równocześnie.
W takim przypadku wymianę ciepła można określić za pomocą wzoru \ref{emisyjnosc_konwekcja}
\begin {equation}
\dot{Q_{całk}}=\alfa_{całk}*\sigma*A_{s}*(T_{s}^4-T_\inf^4)
\label {emisyjnosc_konwekcja}
\end {equation}
gdzie:
\begin {itemize}
\item \alfa_{całk} - całkowity współczynnik wymiany ciepła
\item T_\inf - temperatura powietrza w znacznej odległości
\end {itemize} 
Omówione powyżej procesy podowują, że promieniowanie jest zjawiskiem szczególnie skomplikowanym.

\subsection {Zależność temperatury od dostarczonej energii}
Opisane w podrozdziałach \ref{Przewodnictwo}, \ref{Konwekcja}, \ref{Radiacja} metody obrazują różne sposoby przekazywania energii
między cząsteczkami materii. Po ich poznaniu należy zadać sobie pytanie w jaki sposób ta energia wpływa na temperaturę substancji?
Wielkością reprezentującą zależność między dostarczoną energią a temperaturą substancji jest ciepło właściwe. Ciepło właściwe
jest wielkością charakterystyczną dla materiału i  informuje ono o tym ile ciepła
należy dostarczyć aby ograć 1kg substancji o $1^\circ C$.Opisaną powyżej zależność przedstawia wzór \ref{cieplo_wlasciwe} 
\begin {equation}
c=Q/(m*\Delta T)
\label {cieplo_wlasciwe}  
\end {equation}
gdzie:
\begin {itemize}
\item c - ciepło właściwe $[J/ (kg * K)]$
\item m - masa ciała $[kg]$
\item T - temperatura  $[K]$
\end {itemize}
Znając ilość ciepła dostarczonego do ciała w wyniku procesów przekazywania energii oraz dokonując przekształcenia powyższego wzoru
można w bardzo prosty sposób policzyć zmianę temperatury badanego ciała.


%TODO refactoring. wiecej punktów w litym tekscie 
\chapter{Automaty komórkowe}
\label{cha:Automaty komórkowe}
W rozdziale tym wyjaśniono pojęcie automatu komórkowego, przedstawiono formalną definicję automatów komórkowych oraz ich najczęstsze  zastsowania. Podczas omawiania automatów zamieszczono ogólny algorytm symulacji z wykorzysaniem automatów komórkwoych. Zasadę działania prostych automatów przedstawiono
na przykładzie gry Life. Szczególną uwagę podczas opisu 
 automatów komórkowych poświęcono niehomogenicznym automatom komórkowym, gdyż to one zostały wykorzystane
do symulacji pożaru. 
\section {Definicja automatu komórkowego}
Według definicji \textbf {Ferbera} \textsl{ Automat komórkowy jest dyskretnym, dynamicznych systemem, którego zachowanie jest
całkowicie określone w warunkach lokalnyc relacji.}
Inną definicją, ukazującą automat komórkowy w matematycznym zapisie jest definicja \textbf{Weimara}, przedstawiająca
automat jako czwórkę parametrów:
\begin{Center}
\begin {equation}
\label{def_automatu}
$CellularAutomata=(L,S,N,f)$
\end {equation}
\end{Center}
gdzie
\begin{itemize}
\item L - zbiór komórek tworzących automat
\item S - zbiór stanów, które może przyjmować komórka
\item N - zbiór sąsiadów danej komórki
\item f - funkcja przejścia, która każdą komórkę ze zbioru L przeprowadza  ze stanu $S_i$ w stan $S_(i+1)$. $S_t$ jest jednym ze stanów
		ze zbioru S przyjmowanym przez komórkę w i-tej jednostce czasu, natomist $S_(i+1)$ jest stanem przyjmowanym
		przez komórkę w i+1-ej jednostce czasu na podstawie analizy stanów sąsiadów komórki (N).
\end{itemize}
Innymi słowy, automat komórkowy jest to model matematyczny opisujący siatkę komórek, ich stany oraz reguły przejść
między kolejnymi stanami. Każda komórka z pewnej dyskretnej siatki komórek może przyjmować jeden z określonego zbioru stanów.
W dyskretnych przedziałach czasowych następuje ewolucja komórki, na podstawie jej \textbf{stanu poprzedniego} oraz \textbf{stanu jej sąsiadów}. Ewolucję komórki określa funkcja przejścia. Odpowiedni dobór zbioru stanów oraz funkcji przejść kształtuje cechy automatu, jego dynamikę oraz szybkość.
W wyniku ewolucji komórka przyjmuje kolejny stan ze zbioru S. Ze względu na wielwymiarowość automatów komórkwoych, sąsiedztwa można
definiować w różny sposób. W przypadku dwuwymiarowego automatu najprostszym sąsiedztwem, zwanym też sąsiedztwem von Neumana będzie
zbiór czterech komórek {N,S,E,W} gdzie kolejne literki oznaczają kierunki zgodne z różą wiatrów. W dwuwymiarowych automacie jako sąsiedztwo można przyjmować także zbiór ośmiu komórek (sąsiedztwo \textbf{Moore'a}: cztery wspomniane powyżej kierunki wraz z kierunkami pośrednimi {N, S, E, W, NW, NE, SE, SW}. W trójwymiarowym automacie najprostszym sąsiedztwem jest zbiór wszystkich 
komórek połączonych ścianami z aktualnym sześcianem. Innym sąsiedztwem mogą być wszystkie szcześciany połączone z aktualnie badanym
zarówno ścianami jak i krawędziami. Ważnymi elementami automatu komórkowego jest stan początkowy automatu, czyli z góry określony
stan każdej komórki w chwili 0 (przed pierwszą iteracją algorytmu). W przypadku automatów komórkwoych o ograniczonej siatce konieczne
jest także zdefiniowanie warunków brzegowych. Warunki brzegowe określają reguły przejść dla komórek znajdujących się na brzegach
automatu. W przypadku jednowymiarowej siatki, przykładem warunku brzegowego jest określenie sąsiadów ostatniego $n-tego$ elementu jako 
$n-1$ oraz $1$ - przedostatni oraz pierwszy element.
Poniżej przedstawiono ogólny schemat działania algorytmu komórkowego: \\\\
\begin{center}
\includegraphics{algorytm_automatu_kom}
\end{center}
\\\\
Automaty komórkowe dzięki możliwościom zastąpienia bardzo skomplikowanych wzorów prostymi regułami są szeroko stoswane do modelowania
procesów fizycznych i chemicznych np. symulacje pożarów lasów, budynków, rozprzestrzenianie lekarstw w organizmie ludzkim. 
Innym zastosowaniem jest symulacja zjawisk, które ze względu na swoją naturę, w wyniku braku zjaomości
dokladnych wzorów nie mogą być odzwierciedlone w sposób dokłady. Przykładem takiej symulacji jest na przykład ruch ludzi podczas ucieczki z ewakuowanego budynku.
\section{Klasyfikacja automatów komórkowych}
Od wprowadzenia pojęcia automatu komórkowego, które datuje się na lata 40-ste XX-go wieku powstały różne metody klasyfikacji automatów komórkowych. Jedną z najważniejszych jest podział ze względu na homogeniczność automatu.
\textbf{Homogenicznym automatem komórkowym} nazywamy, automat który spełnia wszystkie postulaty homogeniczności.
Należa do nich:
\begin{itemize}
\item jednakowy zbiór stanów dla każdej komórki
\item jednakowy zbiór reguł dla każdej komórki
\item stały obszar siatki automatu
\item jednakowy schemat określający sąsiadów dla każdej komórki
\item jednakowa metoda aktualizacji wszystkich komórek
\end {itemize}


Jeżeli automat nie spełnia, \textsl {któregokolwiek} z wyżej wymienionych warunków jest klasyfikowany jako \textbf {niehomogeniczny
automat komórkwowy}. 

Najprostszym przykładem klasycznego, homogenicznego automatu komórkowego jest gra \textsl{Life}. 
W grze Life mamy zazwyczaj do czynienia z nieskończoną planszą Każda
z komórek siatki może przyjmować dwa stany: jest żywa lub martwa. Każda z komórek posiada ośmiu sąsiadów - są to komórki przylegające krawędziami i rogami.W grze tej wszystkie komórki zmieniają swój stan, co pewien ustalony odstęp czasu. Nowy stan komórki jest oblicznay
wyłącznie na podstawie jej poprzedniego stanu oraz stanów sąsiadów. Metoda aktualizacji komórek oraz schemat określający nowy 
stan jest identyczny dla wszystkich komórek automatu. Mimo nazwy omawianej symulacji jedynym udziałem człowieka w tej grze jest
ustawienie stanu początkowego komórek.

Innymi sposobami klasyfikacji automatów komórkowych jest podział ze względu na sąsiedztwo (sąsiedztwo Moore'a oraz von Neumana), 
ze względu na wielowymiarowość planszy (jedno-,dwu-,trój-wymiarowa,...), a także ze względu na kształt pojedynczej komórki.
W jednowymiarowym automacie komórką jest odcinek, w dwuwymiarowym najprotszym wariantem jest kwadrat, a w trójwymiarowej sześcian.
Różnie między tymi automatami zostały częściowo omówione w rozdziale poprzendim.


\end {Center}
%TODO dopisać coś wiecej dlaczego mniejsze komórki nie są potrzebne.
% skala  budynku to że nie interesue nas jak dokładnie będzie się palić każde 10cm^3 ale 
% jak ogien będzie się przemieszczał. rozmiar człowieka z perełe
%TODO w sekcji gdzie będzimy sie chwalić wynikami algorytmu napisać o zasadzie zachowania energii
%TODO gestosc materiału mozę się przydać w algorytmie do obliczania masy powietrza po podgrzaniu spr to
\chapter{Algorytm}
\label{cha:Algorytm}
Rozdział przedstawia propozycję algorytmu symulacji rozprzestrzeniania się ognia i dymu podczas pożaru.
Przedstawiony poniżej model jest niehogenicznym automatem komórkowym i jako taki spełnia postulat niehomogeniczności.
W pierwszej części rozdziału przedstawiono wartości parametrów tworzących automat komórkowy. Kolejne podrozdziały 
zawierają szczegółowy opis kluczowych funkcji współtworzących funkcję przejścia.
% Plan rozdziału
% 1. Jaki typ automatu 3D. wielkość - określana przez usera
% 2. Kształt komórek i wielkość komórek
% 3. Sąsiedztwo - nieregularne, inni sąsiedzi przy krawedziach
% 4. Zbiór stanow
% 5. Funkcja przejscia
% 6. W kolejnych podrozdziałach funkcje będące czynnikami funkcji przejścia - przewodnictwo, konwekcja, dym
% 7. Rodzaje komórek. problem wąskich drzwi i jego rozwiązanie
\section {Model automatu}
Zgodnie ze wzorem \ref{def_automatu} będącym istotą przytoczonej w rozdziale \ref{cha:Automaty komórkowe} definicji automatu komórkowego według Weimara jednym z kluczowych elementów jest określenie siatki, czyli powierzchni automatu. W modelu symulaci pożaru w budynku
ze względu na trójwymiarowość zjawiska oraz istotę jego rzeczywistego odtworzenia (możliwość wykorzystania wyników w celu
opracowania modelu ewakuaci osób, badanie przyczyn katastrofy i drogi rozchozenia ognia) nabardziej naturalnym typem automatu 
jest automat \textsl {trójwymiarowy}. 
Rozmiar automatu jest wielkością zmienną, definiowaną przez użytkownika systemu. Pozwala to na odpowiedni dobór ilości komórek w zależności od wielkości rozpatrywanego budynku. Ze względu na fakt, że istotą przedstawionego modelu jest symulacja pożaru wewnątrz budynku rozmiar siatki należy określić tak aby całość
siatki stanowiła budynek.Zastosowany twójwymiarowy automat składa się z szcześciennych komórek o wymiarach $0.5m x 0.5m x 0.5m$. Wielkość komórek została wybrana empirycznie. 
Odpowiedni dobór wielkości komórek automatu ma kluczowy wpływ na jego działanie. Zbyt mała ilość komórek może doprowadzić do utraty
dokładności algorytmu oraz ukazać zniekształcony obraz działania modelu. Zbyt duża ilość elementów powoduje spadek wydajności algorytmu, a w komputerowej realizacji algorytmu oznacza zwiększone zapotrzebowanie na pamięć i moc procesora.
Wybrany na podstawie doświadczeń rozmiar komórki jest najlepszym
kompromisem między między czasem działania a dokładnością modelu.

Typy komórek wchodzących w skład automatu można podzielić na dwie zasadnicze grupy:
\begin{itemize}
\item Ciała stałe
\item Gazy
\end{itemize}
Model symulacji nie uwzględnia interakcji ognia z wodą lub innymi cieczami i nie przedstawia zjawisk fizycznych zachodzących podczas tych interakcji.
W ciałac stałyc funkcje przejścia odzwierciedlają zjawisko przewodnictwa cieplnego. W gazach będących płynami przewodnictwo zastąpione jest konwekcją.
Wszystkie typy komórek poddane są zjawisku radiacji.
Ponadto, komórki reprezentujące ciała stałe dzielimy ze względu na rodzaj materiału z jakiego są stworzone.
Każdy z materiałów posiada zestaw parametrów określających jego właściwości fizyczne:
\begin{itemize}
\item Ciepło właściwe - określa jak zmienia się temperatura ciała w zależności od ilości dostarczonego / oddanego ciepła
\item Gęstość
\item Współczynnik przewodnictwa ciepła - określa zdolność materiału do przewodnictwa ciepła
\item Temperatura zapłonu - określa temperaturę charakterystyczną dla danego materiału, po której osiągnięciu 
	dochodzi do produkcji palnych oparów
\item Palność - określa procentową ilość oparów powstałych po osiągnięciu temperatury zapłonu. 
\end{itemize}
Wyżej wymienione parametry bezpośrednio wpływają na zachowanie funkcji przejścia, powodując zrożnicowane zachowanie komórek w zależności od typu materiału.


Poza różnymi typami komórek, o niehomogeniczności automatu świadczą różne definicje sąsiedztwa.
Ze względu na fakt, że siatka automatu modeluje przestrzeń zamkniętą - budynek - konieczne jest zróżnicowanie sąsiedztwa w środku siatki oraz na jej brzegach.
W zaproponowanym algorytmie wykorzystano zmienną liczbę sąsiadów w zależności od położenia komórki. 
Komórka znajdująca się w środku siatki posiada sześciu sąsiadów. Sąsiadami są komórki przylegające ścianami do aktualnie
rozpatrywanej, co jest twójwymiarowym wariantem sąsiedztwa von Neumana. %TODO Jacek dopisać cos z Wolframa czemu mniej sąsiadów wystarcza, a wiecej
% wcale nie poprafia sytuacji a jest zbednymi oblizeniami
 Rozklad sąsiadów dla komórki znajdującej się w centrum przestrzeni przedstawia rysunek \ref{sasiedzi}
\begin{figure}
\begin {center}
\includegraphics{sasiedztwo.jpg} \\
\caption { Schemat sąsiedztwa}
\label {sasiedzi}
\end {center}
\end{figure}
W przypadku gdy komórka znaduje się na skraju siatki liczba sąsiadów ulega zmniejszeniu. Do zbioru sąsiadów należą komórki, przylegające ścianami do 
rozpatrywanej oraz jednocześnie będące wewnątrz przestrzeni modelu. W skrajnym przypadku, gdy komórka znajdue się w narożniku liczba sąsiadów z sześciu spada
do trzech. Stan komórki brzegowej jest obliczany, podobnie jak w przypadku komórki znajdującej się wewnątrz automatu na podstawie wszystkich jej sąsiadów. Zmniejszona ilość sąsiadów powoduje zmieniony rozkład ich wpływu na nowy stan bieżącej komórki. Waga znaczenia każdej z komórek wzrasta dwukrotnie.
Innymi, alternatywnymi rozwiązaniami sytuacji brzegowych są :
\begin {enumerate}
\item Uznanie za sąsiada ostatniego elementu w danej płaszczyźnie, elementu pierwszego czyli znajdującego się na przeciwległym brzegu. Jest to tak zwane sąsiedztwo
periodyczne. W przypadku symulacji pożaru taki tym sąsiedztwa nie odzwierciedla rzeczywistych interakcji między komórkami w pomieszczeniu. Komórka znajdująca się 
po drugiej stronie budynku nie wpływa bezpośrednio na stan aktualnie rozpatrywanej.
\item Zastosowanie warunków pochłaniających, czyli nadanie komórkom brzegowym z góry określonego, nie uwzględniającego sąsiedztwa stanu. Rozwiązanie to 
powoduje, że komórki brzegowe nie mogą być traktowane jako elementy budynku z rzeczywistymi właściwościami fizycznymi.
\end{enumerate}

Każda z komórek automatu może przyjmować jeden z trzech stanów:
\begin{itemize}
\item Komórka niezapalona
\item Komórka paląca się
\item Dym
\end{itemize}
Poza wyżej wymienionymi trzema stanami, każda komórka psiada wartość swojej aktualnej temperatury, która również w sposób pośredni określa jej stan i ma znaczący
wpływ na wynik funkcji przejścia.


\chapter {Projekt}
Głównym celem projektu jest stworzenie i weryfikacja modelu rozprzestrzeniania się ognia wykorzystując niehomogeniczne automaty
komórkowe, który został opisany w rozdziale \ref{cha:Algorytm} . Nacisk z pracy został położony na opracowanie algorytmu najdokładniej oddającego rzeczywistość.
Aplikacja, o nazwie Sparkle została zarojektowana tak, aby zapewnić użytkownikowi wysoką ergonomię pracy i łatwość nauki.
Podczas projektowania i implementacji szczególna uwaga została poświęcona dalszym możliwościom rozbudy programu oraz testowania 
kolejnych wersji modelu algorytmicznego. Wstęp poniższego rozdziału zawiera ogólne założenia dotyczącego całego projektu. 
Następni w kolejnych podrozdziałąch została opisana architektura aplikacji oraz budowa i funkcjonowanie poszczególnych modułów.
\label{cha:projekt}
\section {Główne zalożenia}
\begin {itemize}
\item Projekt obejmuje zarówno stworzenie i weryfikację modelu rozprzestrzeniania pożaru jak i zaimplementowanie uproszczonej wizualizacji oraz graficznego interfejsu użytkownika (GUI).
\item Interfejs aplikacji powinien umożliwiać edycję budynku w którym przeprowadzana jest symulacja: dodawanie elementów konstrukcji, 
określanie materiałów z których zostały stworzone. 
\item Użytkownik powinien mieć możliwość określenia źródła ognia (miejsca w którym rozpoczyna się pożar).
\item Aplikacja powinna umożliwiać kontrolę nad symulacją: możliwość zatrzymania symulacji, wznowienia, rozpoczęcia od początku.
\item Dodatkowym elementem jest zapis wyników w postaci rozkładu temperatu do pliku, umożliwiający dogłębną analizę rezultatów.
\item Wizualizacja powinna obejmować zarówno rozkład temperaturowy jak i rozprzestrzenianie się rozprzestrzenianie ognia. 
\end {itemize}
\section {Architektura aplikacji}
Aplikacja została podzielona na trzy główne moduły:
\begin{itemize}
\item Controller - odpowiada za interakcję z użytkownikiem, dostarcza GUI umożliwiające kontrolę symulacji
\item Model - przechowuje model symulacji, realizuje algorytmy rozprzestreniania ognia
\item Scene - odpowiada za wizualizację wyników
\end{itemize}
Zależności pomiędzy poszczególnymi komponentami przedstawia diagram komponentów \ref{architektura aplikacji}
\begin{figure}
\begin {center}
\includegraphics{architectureComponentDiagram.jpg} \\
\caption { Architektura aplikacji}
\label {architektura aplikacji}
\end {center}
\end{figure}
Zapewnienie bardzo prostych relacji między modułami pozwala niezależnie rozwijać kolejne części aplikacji, w łatwy
sposób podmieniać i modyfikować ich zachowanie. Inną zaletą zastosowanego modelu jest łatwość
testowania poszczególnych części programu niezależnie.

Przedstawiony model powstał na bazie jednego z najpopularniejszych modeli tworzenia aplikacji wykorzystujących graficzny interfejs użytkownika: Model-View-Controller. Elementem różniącym przedstawiony powyżej model od tradycyjnej architektury Model-View-Controller
jest rozdzielenie elementów GUI pomiędzy dwa moduły:
\begin{itemize}
\item Scene przedstawiającą wyniki aplikacji oraz
\item Controller, który łączy w sobie elementy kontroli i widoku dostarczając użytkownikowi zestaw narzędzi umożliwiających komunikację z aplikacją.
\end {itemize}
\subsection{Moduł Controller}
Moduł kontroler odpowiada za komunikację między użytkownikiem a silnikiem aplikacji (elementem realizującym logikę algorytmiczną). 
Jego podstawowym zadaniem jest dostarczenie łatwego w obsłudze, graficznego interfejsu oraz 
obsluga akcji użytkownika. Wspomniana obsługa akcji obejmuje zarówno zebranie danych od użytkownika, ich przetworzenie
i dostarczenie do modelu jak i pobranie z modelu danych i ich export na zewnątrz aplikacji.
Funkcjonalności dostarczane przez moduł Controller przedstawia diagram przypadków użycia \ref{przypadki uzycia}.
\begin{figure}
\begin {center}
\includegraphics{useCase.jpg} \\
\caption { Przypadki użycia}
\label {przypadki uzycia}
\end {center}
\end{figure}


Controller zbudowany jest z następujących klas.
\begin{itemize}
\item Editor - dostarcza użytkownikowi pole tekstowe umożliwiające edycję budynku. Rejestruje obiekty nasłuchujące, w dalszym ciągu tekstu nazywane Listenerami, których zadaniem jest przechwytywanie akcji i ich obsługa w ramach metod zwanych Handlerami. W przypadku edytora nasłuchiwanie ogranicza się do komend wprowadzanych z klawiatury. Tekst wprowadzony do edytora jest przekazywany klasie EditorParser do dalszego przetworzenia.
\item EditorParser - odpowiada za parsing czyli analizę otrzymanego tekstu oraz wywołanie odpowiednich metod modelu odpowiadających za 
dodanie / usunięcie bloku.
\item MainWindow - reprezentuje główne okno aplikacji. Zawiera edytor tekstowy, scenę, na której przedstawiane są wyniki symulacji oraz 
zestaw paneli menu. Odpowiada za rozmiary i rozmieszczenie zawieranych elementów.
\item MenuPanel - dostarcza zestaw narzędzi (guziów, pól tekstowych, opcji wyboru) umożliwiających sterowanie aplikacją, dodawanie bloków do budynku oraz przeglądanie tekstu pomocy. Rejestruje Listenery obsługujące wyżej wymienione akcje.
\item TopMenu - zawiera elementy umowiżliwiające zapis wyników symulacji do pliku oraz wprowadzenie budynku z pliku.
\item FileChooser - okno wyboru pliku z którego następuje odczyt parametrów budynku lub do którego zapisywane są wyniki.
\item SimulationMgr - odpowiada za przebieg symulacji. Reguluje jej tempo, obsługuje zdarzenia wstrzyania / wznowienia.
\end {itemize}

\subsection {Moduł Scene}
Jedynym zadaniem sceny, zwanej dalej także widokiem jest graficzne przedstawienie dostarczonego modelu.
Modeł Scene nie korzysta z żadnych innych modułów.
Calkowite odizolowanie widoku od innych komponentów umożliwia jego łatwą modyfikację. Pakiet Scene zawiera klasę
\begin {itemize}
\item Scene3D - odpowiedzialną za wizualizację stworzonego budynku oraz aktualizację widoku podczas symulacji.
\end {itemize}
\subsection {Moduł Model}
Komponent model jest silnikiem aplikacji. Zawiera struktury danych oraz algorytmy realizujące symulację w oparciu o niehomogeniczny automat komórkowy. Składa sie z następujących klas:
\begin {itemize}
\item Cell - zawiera zestaw atrybutów opisujących pojedynczą komórkę automatu oraz metod umożliwiających dostęp do nich. Do atrybutów komórki należą:
\begin {itemize}
\item Materiał
\item Masa
\item Temperatura
\item Stan
\item Zawartość palnych oparów
\item Czas pozostały do całkowitego spalenia komórki
\end {itemize}
\item Material - opisuje właściwości materiału z którego może być zbudowana komórka.
Do właściwości materiału należą:
\begin {itemize}
\item Nazwa
\item Kolor
\item Przeźroczystość
\item Czas spalania
\item Ciepło właściwe
\item Przewodnictwo ciepła
\item Temperatura zapłonu
\item Palność - wartość określająca czy materiał jest paliwem czy nie.
\end {itemize}
\item HeatCondctor - odpowiada za obliczenie ilości energii przekazanej za pomocą przewodnictwa między komórkami. Uaktualnia temperaturę komórek w wyniku przekazania energii.  
\item HeatAndVaporsConductorWithConvection - odpowaida za obliczenie ilości energii przekazanej przez konwekcję. Uaktualnia temperaturę komórek w wyniku przekazania energii. Symuluje przepływ oparów, przenoszonych przez prądy konwekcyjne.
\item FireConductor - odpowiada za aktualizację stanu komórki
\item World - odpowiada za stworzenie modelu automatu odzwierciedlającego zbudowany obiekt. Aktualizuje stan automatu przez wywołanie metod klas HeatConducter, HeatAndVaporsConductorWithConvection, FireConductor.
\end {itemize}
Dokładną ilustrację zależności między klasami wchodzącymi w skład omówionych komponentów, z uwzględnieniem elementów implementacyjnych zawierają diagramy zamieszczone w rozdziale \ref{cha:implementacja}. %żeby od razu zawrzeć tam. ze np jakaś klasa dziedziczy po Composite czy użycie swinga itd.
 

% itd.
% \appendix
% \include{dodatekA}
% \include{dodatekB}
% itd.

\bibliography{bibliografia}

\end{document}

