\chapter{Teoria}
\label{cha:Teoria}
Kluczowym elementem, niezbędnym do prawidłowego zamodelowania pożaru
jest zrozumienie czym jest ogien, poznanie zjawisk jakim podlega.Niniejszy rozdział zawiera
krótki wstęp teoretyczny, przedstawiający zjawiska fizyczne niezbędne do zrozumienia istoty 
pożaru i prawidłowego jego zamodelowania.
\section {Czym jest ogień}
\section {Proces spalania}
\section {Propagacja ciepła}
Jak zostało wspomniane w rozdziale \ref{Proces spalania} jednym z czynników niezbędnych
do podtrzymania ognia jest cieło. Ciepło podczas pożaru jest propagowane na trzy różne sposoby:
\begin {itemize}
\item Przewodnictwo
\item Konwekcja
\item Radiacja
\end {itemize}
\subsection {Przewodnictwo}
Przewodnictwo jest procesem, który zachodzi w ciałach stałych, cieczach i gazach. Polega ono na wymianie ciepła 
pomiędzy nierównomiernie ogrzanymi ciałami będącymi w kontakcie. Celem przewodnictwa jest 
osiągnięcie równowagi cieplnej. Podczas przewodnictwa ciepło jest zawsze przenoszone od
ciała o większej temperaturze do ciała o niższej. Zgodnie z zasadą zachowania energi, głoszącą że w układzie 
izolowanym suma wszystkich energii jest stała, ilość energii uzyskanej przez ciało chłodniejsze jest równa
ilości energii oddanej przez cieplejszy obiekt. Energia przenoszona jest wraz z ruchem cząsteczek wewnętrznych.
Nie wszystkie ciała przewodzą ciepło w takim sam sposób.
Zależność między ilością ciepła przewodzonego przez ciało, a jego zmianą temperatury najlepiej opisuje prawo Fouriera.
Przyjmuje ono następującą postać:
$$ q(r,t)=-k*grad T$$
gdzie:
\begin {itemize}
\item k - współczynnik przewodzenia ciepła; jednostka $[W / (m*K)]$ \\
\item T - temperatura $[K]$
\item q - natężenie strumienia ciepła $ [W/(m^2)]$
\end {itemize}
Prawo Fouriera oznacza, że gęstość strumienia ciepła przekazywana w jednostce czasu przez jednostkową powierzchnię 
jest proporcjonalna do gradientu temperatury. Pewnym uproszczeniem prawa Fouriera jest zależność \ref {cieplo_wk}

Do dobrych przewodków należą przede wszystkim:
\begin {itemize}
\item metale
\end {itemize}
Źle przewodzą ciepło:
\begin {itemize}
\item drewno
\item tlen
\item papier
\end {itemize}
