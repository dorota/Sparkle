\chapter {Projekt}
\label{cha:projekt}
\section {Główne zalożenia}
\begin {itemize}
\item Głównym celem projektu jest stworzenie i weryfikacja modelu rozprzestrzeniania się ognia wykorzystując niehomogeniczne automaty
komórkowe. Nacisk z pracy został położony na opracowanie algorytmu najdokładniej oddającego rzeczywistość.
\item Projekt obejmuje także swtworzenie uproszczonej wizualizacji symulacji oraz graficznego interfejsu użytkownika (GUI).
\item Interfejs aplikacji powinien umożliwiać edycję budynku w którym przeprowadzana jest symulacja: dodawanie elementów konstrukcji, 
określanie materiałów z których zostały stworzone. 
\item Użytkownik powinien mieć możliwość określenia źródła ognia: zarówno jego miejsca jak i temperatury początkowej.
\item Aplikacja powinna umożliwiać także kontrolę nad symulacją: możliwość zatrzymania symulacji, wznowienia, rozpoczęcia od początku,
a także dostosowanie tempa symulacji umożliwiającego obserwację zjawisk fizycznych.
\end {itemize}
\section {Architektura aplikacji}
% w bibliografi powinno byc coś o tej architekturze i def. aktywnego modelu
Aplikacja została zaprojektowana zgodnie z architekturą Model-View-Controller.
Poniższy schemat przedstawia typowy przykład architektury Model-View-Controller:\\
\includegraphics{MVC.jpg} \\
W omawianej pracy została zaimplementowana pewna odmiana typowej architektury Model-Widok-Kontroler.
Najodpowiedniej przedstawia ją poniższy rysunek:\\
\includegraphics{modifiedMVC.jpg}\\
Kontroler odpowiada za pobranie danych od użytkownika, ich przetworzenie oraz dostarczenie do modelu.
Został zastosowany przypadek aktywnego modelu, który zgodnie z definicją potrafi zmieniać swój stan 
bez względu na akcje wykonywane przez użytkownika. W projekcie symulacji pożaru aktywność modelu polega na 
wykonywaniu pętli symulacji, związanych z nią obliczeń, powiadamianiu widoku o zachodzących zmianach oraz
końcu symulacji. Widok odpowiada jedynie za prezentację wyników symulacji.
Wybrana architektura umożliwia elastyczny rozwój aplikacji. Wprowadzony podział na trzy odrębne moduły pozwala
na nieograniczone zmiany w każdym z nich, nie powodując konieczności zmian innych części aplikacji.
Inną zaletą separacji jest łatwość testowania poszcególnych modułów osobno. 
\section {Moduły}
\section {Obiekty}
